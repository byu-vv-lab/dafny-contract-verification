\documentclass{article}

% Language setting
% Replace `english' with e.g. `spanish' to change the document language
\usepackage[english]{babel}
\usepackage{listings}

% Set page size and margins
% Replace `letterpaper' with`a4paper' for UK/EU standard size
\usepackage[letterpaper,top=2cm,bottom=2cm,left=3cm,right=3cm,marginparwidth=1.75cm]{geometry}

% Useful packages
\usepackage{amsmath}
\usepackage{graphicx}
\usepackage{algpseudocode}
\usepackage[colorlinks=true, allcolors=blue]{hyperref}

\usepackage[utf8]{inputenc}
\usepackage[square,numbers]{natbib}
\usepackage{xcolor}

\usepackage{tabularx}
\usepackage{caption}

\newcommand{\figref}[1]{Fig.~\ref{#1}}
\newcommand{\secref}[1]{Sec.~\ref{#1}}
\newcommand{\tabref}[1]{Table~\ref{#1}}

\lstdefinelanguage{dafny}{
  literate=%
    {<=}{$\le$}1
    {>=}{$\ge$}1
    ,
  sensitive=true,  % case sensitive
  morecomment=[l]{//},
  morecomment=[s]{/*}{*/},
  morecomment=[s]{\{:}{\}},
  morestring=[b]",
  numbers=none,
  firstnumber=0,
  numberstyle=\tiny,
  stepnumber=5,
  commentstyle=\itshape,
  keywordstyle=\bfseries,
  ndkeywordstyle=\bfseries,
  basicstyle=\ttfamily,
    morekeywords={class,datatype,codatatype,newtype,type,iterator,trait,extends,
      bool,char,nat,int,real,object,set,iset,multiset,seq,string,map,imap,array,array2,array3,
      bv0,bv1,bv2,bv3,bv4,bv5,bv6,bv7,bv8,bv12,bv16,bv20,bv24,bv32,bv48,bv64,bv128,
      const,
      function,predicate,copredicate,inductive,
      ghost,var,static,protected,refines,
      method,lemma,constructor,colemma,twostate,
      returns,yields,abstract,module,import,default,opened,as,in,
      requires,modifies,ensures,reads,decreases,include,provides,reveals,witness,
      % expressions
      match,case,false,true,null,old,fresh,allocated,unchanged,this,
      % statements
      assert,by,assume,print,new,if,then,else,while,invariant,break,label,return,yield,
      where,calc,modify
      }
}

\newif\ifcomments
\commentstrue
\newcommand{\egm}[1]{\ifcomments\textcolor{orange}{egm: #1}\fi}
\newcommand{\cass}[1]{\ifcomments\textcolor{blue}{cass: #1}\fi}

\bibliographystyle{abbrvnat}

\title{Software Contract Quality}
\author{Cassidy Waldrip}

\begin{document}
\maketitle

\begin{abstract}

Software contracts, or specifications, define the intended behavior of systems. They are often used to check
that an implementation of a system is a safe substitution for the contract that defines it, meaning that the implementation could replace the contract while 
maintaining the intended behavior. 
Writing contracts is error-prone and there are limited tools for detecting their faults. This proposal presents four common
pitfalls in software contracts and describes a plan to detect those pitfalls. The pitfalls are contradictions, vacuity, unconstrained outputs, and 
redundancy.
It also outlines the blueprint for a tool built for the Dafny programming language that would detect and mitigate the pitfalls. Dafny is a language that allows 
developers to create logical contracts to guide the implementation of their programs, and uses an SMT solver to check that the implementation is a 
safe substitution for the contract. The ideas
and algorithms presented can be extended to other contract-based languages.

\end{abstract}

\section{Introduction}

There is a growing importance of specifications in high-assurance systems – knowing what should be happening
in the system versus what is actually happening is crucial, with the obvious goal being agreement
between the two. Specification-aware languages such as Dafny \cite{leino2010dafny} and Alloy \cite{jackson2002alloy} are also becoming more prevalent in
software engineering, both for contract-driven development and for increasing confidence in existing systems.

Specifications, or software contracts, are used to describe the intended behavior of a program. They are used
to specify both the inputs and outputs of a system in the form of preconditions and postconditions. Preconditions
describe the required state of the system at the beginning of a method as well as the domain of the inputs,
while postconditions describe the state of the system after the method has executed. Software contracts
are used by languages such as Dafny to verify that implementations are a safe substitution for their associated contracts. 
The notion of a safe substitution comes from the Liskov Substitution Principle \cite{liskov1987keynote}, which states that 
an object of a superclass should be replaceable with an object of any of its subclasses, without breaking the application. 
This principle is upheld by making sure that the subclass's preconditions are no {\it stronger} than the superclass's 
preconditions, while its postconditions are no {\it weaker} than the superclass's postconditions. Through the lens of 
contracts and implementations, the contract is the superclass - it defines behavior that should {\it always} happen. The 
implementation of the contract is a subclass. Dafny proves that the relationship between the two does not violate the 
Liskov Substitution Principle, meaning that the implementation could safely replace the contract, and guarantee the 
same behavior. However, a guarantee of this form can be misleading - though one can be sure that the implementation is a 
safe substitution for the contract, there is no guarantee that the contract itself is without defects.

Writing correct contracts can be as error prone as writing correct implementations, so a proof of safe substitution 
between the two can be meaningless if there are issues in the specification. There are many common pitfalls
that can easily go undetected, and the consequences of these pitfalls can often completely void the utility of
a contract. Even worse, a bad contract gone undetected can create a false sense of confidence in a system,
because contracts are often tied to proofs. Though a system “proves out”, it only proves out relative to the
specifications. Faulty specifications lead to implementations that do not have the intended behavior, so developers 
need to be extremely confident that the former are correct. 

This work will focus on mitigating four common pitfalls of software contracts: contradictions, vacuity, unconstrained
outputs, and redundancy. A contract is a contradiction if it is not realizable, meaning that it 
contains requirements that are impossible to satisfy. Contracts with vacuity are similar, and contain components that are contradictions, 
but the contradictions appear as the left-hand side of an implication. These types of contradictions create a technically satisfiable contract, as 
a false premise always leads to a true conclusion. However, there is no case where a developer would want to specify a condition that could 
never be met. Outputs are unconstrained when it is possible, given certain inputs, to end up with outputs that are in a domain that is not specified.
The implications of an unconstrained output can be large - it means that there is a section of an output that is
completely irrelevant in verification. Finally, A contract is redundant when there are unnecessary components. Contracts
with redundancy can become convoluted, making them more difficult to read and understand, and therefore more
error-prone.

We propose an implementation-agnostic SMT-based solution to the three mentioned pitfalls. The solution is implementation-agnostic in the sense
that there is no reliance on any associated implementation to analyze a contract. The idea of "design by contract" \cite{meyer1992applying} is that the 
ideal way to create a software system is to first understand {\it what} it will do, and then figure out {\it how} to do it. Writing a contract and 
any appropriate proofs is the first step, implementing the intent of the contract follows. This practice allows designers to reason in the abstract 
before working out how the actual computation takes place, as the properties and key components of the system are known prior to implementation. 
Verifying contracts independently of any implementation keeps specification and implementation issues separate. 

\section{Problem}

This work will provide a means of checking the quality of logical software contracts in order to increase assurance
that implementations and proofs of agreement between contracts and implementations are correct. It will focus on
examining the contracts independently of their corresponding implementations, and by doing so will allow developers
to have a sanity check that does not require a model or implementation to be available. Logical contracts will be
checked for three main pitfalls - contradiction, unconstrained outputs, and redundancy. The contract quality checker
will be built for the Dafny programming language, but the ideas presented are easily extendable to other languages
and frameworks that use software contracts. 

\section{Related Work}

Creating correct software contracts is a well-known issue \cite{rozier2016specification} \cite{kupferman2006sanity}.
Barnat et al. \cite{barnat2012checking} \cite{barnat2016analysing} proposed a method for checking contract quality using model checking,
both for contracts with associated models and for standalone contracts. The work done for model-free sanity checking is
similar to the goals of this thesis, though that method transforms LTL to automata, rather than using an SMT solver. The approach
has the inherent limitation that transforming LTL formulas into models can become complex for large systems.

Filipovikj et al. \cite{filipovikj2017smt} propose a procedure that uses the Z3 SMT solver \cite{de2008z3} to check software contracts
for contradictions. This procedure is different from ours because it is intended for contracts that are written in
natural language, as opposed to formal contracts used in verification languages. Much of the work is to translate
natural language to constraints that can be passed to Z3. Their tool does not check contracts
for unconstrained output or redundancy. 

Finding an irredundant set of requirements is a complex problem, as shown through the lens of finding the simplest form 
of a proposition \cite{umans2006complexity}. Logic minimization has been studied for propositions that are in sum-of-product form, which is a 
disjunction of conjunctions \cite{coudert1994two}. SMT solvers take constraints in conjunctive normal form, or a conjunction of disjunctions, so to use these 
techniques, the contracts would have to be converted to sum-of-product form. Yang et al. \cite{yang2000bds} proposed a solution that uses binary 
decision diagram \cite{akers1978functional} variable reordering to 
find simplifications of boolean logic. More recent work leverages e-graphs, which are structures that represent congruence relations over expressions \cite{nelson1980fast}. For example, a tool called egg \cite{willsey2020egg} uses e-graphs to collect many different versions of the same expression, and determines the most efficient version. None of these techniques have been used for software contract redundancy checking, as far as we know. 
\cass{Added stuff about egg - it was interesting! Let me know if I misunderstood anything.}

Work has been done to determine when components of a system are sufficiently covered. Jaffe et al. \cite{jaffe1989completeness}
looked at different definitions of contract coverage, especially in a black-box environment. Their work asserts that "completeness"
with respect to outputs in a contract means that each output must be fully specified. This work will use this definition of
completeness to check for unconstrained outputs.

As we are using an SMT solver to check for various defects, a known limitation of SMT solvers should be mentioned. When an SMT problem can be satisfied, it is easy 
to check that the satisfying assignment is valid. However, when a solver says that a problem is unsatisfiable, there is no real proof certificate that the problem 
really cannot be satisfied. To mitigate this issue, some SMT solvers, including Z3, produce something called an UNSAT core \cite{guthmann2016minimal}. An UNSAT core is still not a 
direct proof that the problem is indeed unsatisfiable, but it provides the portion of the problem that the SMT solver believes makes it unsatisfiable. This provides extra information 
that can be used to increase confidence that the SMT solver is correct. 

Our work will be implemented for Dafny. There are other languages and frameworks that support software contracts. Eiffel \cite{meyer1988eiffel} is a language that 
is based on the idea of design-by-contract, and supports creating contracts for its methods. The AGREE model checker allows developers to write contracts for their 
software, and then uses model checking to verify whether or not the system upholds the requirements \cite{mercer2023synthesizing}. Other languages with similar 
support for contracts are Alloy \cite{jackson2002alloy}, F\# \cite{syme2010f}, and JML \cite{leavens1998jml}. 

\section{Contract Pitfalls}

\subsection{Contradictions}

A contract is a contradiction if there are requirements or assertions that can never be simultaneously be true, no matter the values assigned 
to its variables. Conversely, a contract is consistent if there does exists a satisfying assignment. More formally, a
contract \(\Phi\) is made up of specification clauses \(\Phi = \{\varphi_{1}, \varphi_{2}, ..., \varphi_{n}\}\).
A specification clause is made up of logical expressions. A contract is consistent if there exists a way to satisfy
\(\varphi_{1} \land \varphi_{2} \land ... \land \varphi_{n} \). A contract is a contradiction
if \(\varphi_{1} \land \varphi_{2} \land ... \land \varphi_{n} \implies false\). \tabref{tab:warn} below shows the warning signs of contradictions that Dafny gives in different 
contexts. There are never warnings without an implementation, which is the focus of this work. However, there are some cases where a developer would get some clues about 
faults in the contract. 

\begin{center}
\begin{tabularx}{0.9\textwidth} { 
  | >{\raggedright\arraybackslash}X 
  | >{\raggedright\arraybackslash}X 
  | >{\centering\arraybackslash}X 
  | >{\raggedright\arraybackslash}X 
  | >{\raggedleft\arraybackslash}X | }
     \hline
      & \textbf{Contract alone} & \textbf{Contract with implementation} & \textbf{Calling context} \\
     \hline
     \textbf{Preconditions} & No warnings  & No warnings & No valid inputs \\
     \hline
     \textbf{Postconditions}  & No warnings  & Violated postcondition & No warning \\
    \hline
\end{tabularx}
\captionof{table}{Contradictions in various scenarios and their associated warning signs.}
\label{tab:warn}
\end{center}

\egm{Add text to introduce the table and explain what the reader is looking it.}
\cass{Done}

As previously stated, we want to find contradictions in a contract {\it before} writing an implementation or calling the associated method. 
At this point, Dafny will not give any warnings of contradictions without an associated implementation, whether in the preconditions or postconditions. 

 \paragraph{Requirements with contradictions}

 A contradiction in a requirement boils down to requiring
 \(false\). With a false pretense, any assertion can be proven to be vacuously true. This is a tricky case
 where an SMT solver will not detect an issue, even with an associated implementation, because \(false \implies true\) is valid logically. The implementation
 will appear to prove out with no issue whatsoever. However, when 
 a precondition to a method is a contradiction, then the method cannot be called in any context because the precondition is 
 always violated, which can be a teller that something is amiss.

 In the following example, it is quite simple to see the contradiction in the assertions. When systems are large and more
convoluted, contradictions can be more difficult to spot. The simpleness of the example is to get to the heart of the matter.

\egm{Create a listings environment for Dafny and add in any additional Dafny keywords. See \href{https://github.com/ericmercer/SPIN-bpmn-cwp-verification-paper.git}{my spin paper} where I define an environment for Promela. Also, create the \emph{figref}, \emph{tabref}, and \emph{secref} comments for referring to figures, tables, and sections respectively. Example of that should be the paper as well.}
\cass{Someone posted a language definition for dafny for latex online and I edited it...but I don't know why the keywords aren't getting bolded! Idk, it just looks weird, and it seems like none of the 
settings I'm trying to set are working. I also added the referring stuff}

Consider the following example:

\begin{lstlisting}[language=dafny]

method BinarySearch(a: array<int>, key: int) returns (index: int)
    requires forall i, j | 0 <= i < a.Length && 0 <= j < a.Length
        :: i <= j ==> a[i] < a[j]
    ensures 0 <= index < a.Length ==> a[index] == key
    ensures index < 0 ==> forall k :: 0 <= k < a.Length ==> a[k] != key

\end{lstlisting}

The \(requires\) clause above has an error. The intent of the clause is to require a sorted array \(a\) as input.
However, the claim that \(i <= j \implies a[i] < a[j]\) is false. If the indices are the same, then \(a\) at that
index has to be the same. Because the relationship between requires and ensures is an implication, this false requirement
makes any ensures clause provable. As indicated in the table, Dafny currently will not catch this error, with or without an implementation attached to
the contract, but in trying to call the method, one would find that there are no suitable inputs.  

\paragraph{Assertions with contradictions}

For contradictions found in postconditions, It is worth noting that 
an SMT solver {\it will} detect the mismatch between the contract and the implementation and report an error. However, 
the only knowledge the programmer comes away with is that the contract and the implementation do not agree in some way. They are none the wiser 
about what is causing the issue – is it an issue with their implementation, their contract or both? In a calling context, there is no warning 
whatsoever, even with an unsuitable implementation, and anything will prove out in the context where the bad method is called. 

\begin{lstlisting}[language=dafny]

method Contradiction(a : bool, b : bool) returns (result : bool)
    requires a && b
    ensures result
    ensures a ==> !result

\end{lstlisting}

We are trying to ensure that \(result\) is both true and untrue. This is a contradiction. All clauses must be satisfiable
at the same time for a contract to be consistent. As stated previously, with an implementation attached to this contract,
Dafny would catch the issue and report that the implementation does not agree with the contract. What Dafny would not
tell the developer is that the implementation can {\it never} satisfy the specification. In systems with complicated
contracts, a developer could spend time trying to fix an implementation that has no hope of satisfying the contradictory
contract. If the method were to be called in another method, there would be no warnings, and everything beyond the call-site 
would prove out. For example, after a call to \emph{Contradiction(true, true)} Dafny proves \emph{assert(result)} and it proves \emph{assert(!result)}. 

\egm{For example, after a call to \emph{PostconditionContradiction(true, true)} Dafny proves \emph{assert(result)} and it proves \emph{assert(!result)}. Change the method name to something shorter too so it fits on the line.} 
\cass{Done!}

\subsection{Vacuity}

An assertion is vacuously true when it is always true, regardless of inputs. 

To demonstrate vacuity, consider the following example:

\begin{lstlisting}[language=dafny]

method Vacuity(a : bool, b : bool) returns (result : bool)
    requires a && b
    ensures !a ==> !result

\end{lstlisting}

The left side of the implication in the \(ensures\) clause above will always evaluate to {\it false}, as {\it a} is required to be true. 
Therefore, because \(!a == false\) and \(false \implies true\), the ensures clause will always be satisfied! This is never the intent of the 
developer - why would anyone write an implication that has no effect? In a more complicated environment, vacuous clauses can be difficult to 
spot, and there is currently no way for Dafny to detect them, with or without an implementation attached to
the contract.

\subsection{Unconstrained outputs}

When writing a contract, the whole point is to be able to talk about the behavior of the system being specified.
Every system has inputs (whether implicit or explicit) and outputs, and specifications tell us what these pieces need
to look like. Let’s look at the idea of unconstrained outputs with an example:

\begin{lstlisting}[language=dafny]

method AbsoluteValue(n : int) returns (result : int)
    ensures n < 0 ==> result == -1 * n

\end{lstlisting}

It is easy to see that the case where the input \(n < 0\) is covered - we know exactly what {\it result} should look like in that case. 
What about the case where \(n \geq 0\)? We know nothing about {\it result}, so anything goes. Unconstrained outputs should be detected before 
writing an implementation, so that the implementation can be fully based off of the contract. 

To illustrate the issues that can happen when outputs are not fully constrained, let's look at an example with an implementation attached to the 
above contract.

\begin{lstlisting}[language=dafny]

method AbsoluteValue(n : int) returns (result : int)
    ensures n < 0 ==> result == -1 * n
{
    if (n < 0) {
        return -1 * n;
    } else {
        return n;
    }
}

\end{lstlisting}

We can easily see that this implementation of a method that returns the absolute value of an integer is correct. A developer may be
tempted to only ensure something about the case that changes the value of \(n\), as we see in the contract. However, in the case that \(n >= 0\), the value
of \(result\) is totally unconstrained. If we changed the implementation to return a hard-coded value for that case, everything
would still prove out:

\begin{lstlisting}[language=dafny]

method AbsoluteValue(n : int) returns (result : int)
    ensures n < 0 ==> result == -1 * n
{
    if (n < 0) {
        return -1 * n;
    } else {
        return 42;
    }
}

\end{lstlisting}

Not accounting for every possible value of outputs causes portions of an implementation to not affect verification at all. In
most cases, it is important to understand the conditions lead to all possible outputs. This is another case where a
developer could be led to believe that an implementation that proves out is correct, when in reality, the specification is
simply not strong enough.

\subsection{Redundancy}

A software contract is redundant if there are unnecessary components in it. More formally, given a contract
\(\Phi = \{\varphi_{1}, \varphi_{2}, ..., \varphi_{n}\}\), if \(\{\varphi_{2} \land ... \land \varphi_{n}\} \implies \varphi_{1}\), then
\(\varphi_{1}\) is redundant - it is already implied by the other formulae!

While redundancy does not cause direct errors, it can lead to convoluted specifications that are more error prone. Specifications should be as
clear and simple as possible, so the removal of redundant properties allows for more streamlined specifications and takes away some chance
for misunderstanding or error.

Consider the following example:

\begin{lstlisting}[language=dafny]

method Redundant(x : int) returns (result : int)
    requires 0 <= x
    requires x != -1

\end{lstlisting}

In this simple example, we can see that the developer wants to specify that the input must be between 0 and 100, and cannot be
-1. The \(x \neq -1\) is redundant. \(0 \leq x < 100 \implies x \neq -1\) already.
\egm{How does the first requires exclude 50? I commonly see things like $m \le n \wedge m < n$.}
\cass{Took out the 50.}
\egm{Create a single section for all the above contract issues and make each type of issue a subsection. You can use subsubsection or \textbackslash paragraph if needed.}
\cass{Done.}
\section{Approach}

We will use the Z3 SMT solver as the main tool for detecting each above pitfall. As previosly stated,
this work will be implemented for Dafny, but the techniques used could easily be used elsewhere.

\subsection{Detecting contradictions and vacuity}

Detecting contradictions in specifications is as simple as determining whether the contract as a whole is satisfiable.
We need only to take the conjunction of each subformula, and ask Z3 if there is a satifying assignment. If Z3 tells us that the
formula is unsatisfiable, we know that we have a contradiction. However, we want more information than just that. We
want to know what parts of the contract are contradictions with other parts. This set of contradicting formulas could be
nontrivial, as there could exisit multiple subformulas that do not agree with other subformulas. We want to report all
such combinations to the developer.

In order to acheive this, we draw from Barnat's \cite{barnat2016analysing} work on contradiction detection.
To find the combinations of consistent formulae and the combinations of contradicting formulae in a contract
\(\Phi = \{\varphi_{1}, \varphi_{2}, ..., \varphi_{n}\}\), we start with the set \(\{\varphi_{1} \land \varphi_{2} \land ... \land \varphi_{n}\} \)
and the sets that contain each individual formula: \(\{\varphi_{1}\}, \{\varphi_{2}\}, ... \{\varphi_{n}\} \), and check
the consistency of each set. If any of the individual formulae are contradictions alone, then \(\Phi\) will also be a contradiction.
We then go through a processes of removing formulae from \(\Phi\), and adding formulae to consistent individual sets.

We will also check for contradictions in clauses that have implications, to discover any vacuity in requirements.

\subsection{Detecting unconstrained output}

Unconstrained output has to do with the domain of the output variables of a method. If a contract describes the entire
domain of the variable at hand, then every possible value of that variable is accounted for. The process of detecting
unconstrained output can also be accomplished by using an SMT solver. Jaffe et al. \cite{jaffe1989completeness} discuss how
an output is \emph{complete}, or in our terms, constrained, if the disjunction of all constraints on that output is a
tautology. For example, if there is a clause that ensures something about an output variable {\it o} if \(x < 5\), in order
for the output to be fully constrained, we would need a constraint, or set of constraints, that cover \(x \geq 5\) and ensure 
something about {\it o}. Thus \(x < 5 \lor x \geq 5\), which always evaluates to true.

It is important to note that the requirements on the inputs can change the domain that needs to be covered. If, for example, 
an input {\it x} is constrained by the precondition {\it requires} \(x > 0\), then the only constraints on {\it x} that need to 
be covered are those where \(x > 0\). So, the constraints on the output must be a tautology under the precondition: 
\((x > 0) \land (x < 5 \lor x \geq 5)\).

Our approach for detecting unconstrained outputs will be based on this insight:
\break

\begin{algorithmic}

\ForAll{$o \in \mathcal Outputs $}
    \State $relevantClauses \gets c_{1}, c_{2}, ..., c_{n}$ \Comment{Relevant clauses are those that mention o}
    \State $requirements \gets r_{1}, r_{2}, ..., r_{n}$
    \If{$r_{1} \land r_{2} \land ... \land r_{n} \land (c_{1} \lor c_{2} \lor ... \lor c_{n})$ is true}
        \State $result \gets constrained$
    \Else
        \State $result \gets unconstrained$
    \EndIf
\EndFor

\end{algorithmic}

Sending the disjunction of the relevant clauses to Z3 will tell us whether each output is sufficiently constrained. 
We are also interested in finding a way to return the part of the domain for each unconstrained output that
is not covered. 
\egm{OK. Interesting. So if the ensures form a tautalogy then the output is constrained under every input? How do the pre-conditions factor in? It would have to be a tautology under the pre-condition constraints. Right? How is that logically stated?}
\cass{Does what I wrote make sense? Anding the preconditions on to the disjunction of relevant clauses makes it so that we are only checking the relevant domain. I added that to the pseudocode}

\subsection{Detecting redundancy}

Redundancy also draws from Barnat's work. Given a contract \(\Phi = \{\varphi_{1}, \varphi_{2}, ..., \varphi_{n}\}\),
\(\varphi_{1}\) is not a redundant formula if \(\{\not \varphi_{1} \land \varphi_{2} \land ... \land \varphi_{n}\}\)
is satisfiable, because \(\{\varphi_{2} \land ... \land \varphi_{n}\} \not\implies \varphi_{1}\). So, we can check
each formula to see if it is redundant with respect to the rest of the formulae. 

It may also be possible to use logic minimization tools to detect redundancy, and produce a simplified set of requirements. That would require 
converting the contract into a boolean predicate and sending that predicate to a logic minimizer. Once minimized, it would have to be reified back 
into its original form. Similarly, an e-graph may help to find rewrites of the contract and to determine the minimal version. 

\section{Deliverables}

The main deliverable of this work will be a set of methods that use an SMT solver to determine the quality of software
contracts. The thesis will provide algorithms, as well as the specifics of the implementation in Dafny in order to make
extensions of this work accessible in the future. These methods will be implemented for Dafny, but could easily be
applied to other specification-based languages. The tool we create will be an option in Dafny that a developer can
use to check their specifications of any given function or method. Any potential issues will be reported back to the
developer.

\section{Timeline}

All of the research and creation of deliverables will be completed during the months of May-August 2023. The writing of the thesis and defense will happen between September-December 2023.

% \bibliographystyle{alpha}
% \bibliography{sample}
% \bibliographystyle{agsm}
\bibliography{References}

\end{document}