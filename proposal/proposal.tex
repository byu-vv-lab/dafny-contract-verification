\documentclass{article}

% Language setting
% Replace `english' with e.g. `spanish' to change the document language
\usepackage[english]{babel}
\usepackage{listings}

% Set page size and margins
% Replace `letterpaper' with`a4paper' for UK/EU standard size
\usepackage[letterpaper,top=2cm,bottom=2cm,left=3cm,right=3cm,marginparwidth=1.75cm]{geometry}

% Useful packages
\usepackage{amsmath}
\usepackage{graphicx}
\usepackage{algpseudocode}
\usepackage[colorlinks=true, allcolors=blue]{hyperref}

\usepackage[utf8]{inputenc}
\usepackage[square,numbers]{natbib}
\bibliographystyle{abbrvnat}

\title{Software Contract Quality}
\author{Cassidy Waldrip}

\begin{document}
\maketitle

\begin{abstract}

Software contracts, or specifications, define the intended behavior of systems. They are often used to check
conformance and correctness of implementations, and therefore must be correct if they are to be used effectively.
Writing contracts is error-prone and there are few tools for their debugging. This proposal presents three common
pitfalls in software contracts and describes a plan to detect those pitfalls. It also outlines the
blueprint for a tool built for the Dafny programming language that would detect and mitigate the pitfalls. The ideas
and algorithms presented can be extended to other contract-based languages.

\end{abstract}

\section{Introduction}

There is a growing importance of specifications in high-assurance systems – knowing what should be happening
in the system versus what is actually happening is crucial, with the obvious goal being agreement
between the two. Specification-aware languages such as Dafny \cite{dafny} and <<>> are also becoming more prevalent in
software engineering, both for contract-driven development and for increasing confidence in existing systems.

Specifications, or software contracts, are used to describe the intended behavior of a program. They are used
to specify both the inputs and outputs of a system in the form of preconditions and postconditions. Preconditions
describe the required state of the system at the beginning of a method as well as the domain of the inputs,
while postconditions describe the state of the system after the method has executed. Software contracts
are used by languages such as Dafny to verify that implementations match their intended behavior, by proving
that a program’s contract and implementation agree. However, a proof of this form can be misleading - though
one can be sure that the implementation adheres to the contract, there is no guarantee that the contract itself
is correct.

Writing correct contracts can be as error prone as writing correct implementations, so a proof of adherence
between the two can be meaningless if there are issues in the specification. There are many common pitfalls
that can easily go undetected, and the consequences of these pitfalls can often completely void the utility of
a contract. Even worse, a bad contract gone undetected can create a false sense of confidence in a system,
because contracts are often tied to proofs. Though a system “proves out”, it only proves out relative to the
specifications. Faulty specifications lead to faulty implementations, so developers need to be extremely confident
that the former are correct.

This work will focus on mitigating three common pitfalls of software contracts: inconsistency, unconstrained
outputs, and redundancy. A contract is consistent if all of its components are realizable, meaning that the contract
does not contain contradicting requirements. Left unchecked, inconsistencies can make a contract impossible to satisfy
or cause properties to be vacuously true, as a false premise always leads to a true conclusion. Outputs are
unconstrained when it is possible, given certain inputs, to end up with outputs that are in a domain that is not specified.
The implications of an unconstrained output can be large - it means that there is a section of an output that is
completely irrelevant in verification. Finally, A contract is redundant when there are unnecessary components. Contracts
with redundancy can become convoluted, making them more difficult to debug and understand, and therefore more
error-prone.

We propose a black-box SMT-based solution to the three mentioned pitfalls. The solution is black-box in the sense
that it does not require associated implementations for the contracts it analyzes. This approach makes the ideas and tools
more extendable, as the only inputs are logical constraints that can be sent to an SMT-solver. It also allows developers
to sanity check their specifications {\it before} linking them to an implementation, so as not to convolute the issues
that arise from specifications alone, or the linking of the specifications and the implementation.

\section{Problem}

This work will provide a means of checking the quality of logical software contracts in order to increase assurance
that implementations and proofs of agreement between contracts and implementations are correct. It will focus on
examining the contracts independently of their corresponding implementations, and by doing so will allow developers
to have a sanity check that does not require a model or implementation to be available. Logical contracts will be
checked for three main pitfalls - inconsistency, unconstrained outputs, and redundancy. The contract quality checker
will be built for the Dafny programming language, but the ideas presented are easily extendable to other languages
and frameworks that use software contracts.

\section{Related Work}

Creating correct software contracts is a well-known issue \cite{rozier2016specification} \cite{kupferman2006sanity}.
Barnat et al. \cite{barnat2012checking} \cite{barnat2016analysing} proposed a method for checking contract quality using model checking,
both for contracts with associated models and for standalone contracts. The work done for model-free sanity checking is
similar to the goals of this thesis, though that method transforms LTL to automata, rather than using an SMT solver. The approach
has the inherent limitation that transforming LTL formulas into models can become complex for large systems.

Filipovikj et al. \cite{filipovikj2017smt} propose a procedure that uses Z3 \cite{de2008z3} to check software contracts
for inconsistencies. This procedure is different from ours because it is intended for contracts that are written in
natural language, as opposed to formal contracts used in verification languages. Much of the work is to translate
natural language to constraints that can be passed to Z3. Their tool does not check contracts
for unconstrained output or redundancy.

Work has been done to determine when components of a system are sufficiently covered. Jaffe et al. \cite{jaffe1989completeness}
looked at different definitions of contract coverage, especially in a black-box environment. Their work asserts that "completeness"
with respect to outputs in a contract means that each output must be fully specified. This work will use this definition of
completeness to check for unconstrained outputs.

\section{Inconsistency}

A contract is inconsistent if there are assertions or requirements that contradict one another. More formally, a
contract \(\Phi\) is made up of specification clauses \(\Phi = \{\varphi_{1}, \varphi_{2}, ..., \varphi_{n}\}\).
A specification clause is made up of logical expressions. A contract is consistent if there exists a way to satisfy
\(\varphi_{1} \land \varphi_{2} \land ... \land \varphi_{n} \). Conversly, a contract is inconsisten
if \(\varphi_{1} \land \varphi_{2} \land ... \land \varphi_{n} \implies false\).

Inconsistent assertions about the end state of a program can never be met by the associated implementation.
This may seem unimportant, as an SMT solver will detect the mismatch between the contract and the implementation
and report an error. However, the only knowledge the programmer comes away with is that the contract and the
implementation do not agree. They are none the wiser about what is causing the issue – is it an issue
with their implementation, their contract or both?

Inconsistent requirements are arguably even more dangerous. An inconsistent requirement boils down to requiring
 \(false\). With a false pretense, any assertion can be proven to be vacuously true. This is the tricky case
 where an an SMT solver will not catch the issue, because \(false \implies true\) is valid logically. The implementation
 will appear to prove out with no issue whatsoever. This issue also arises in assertions that include implications: if
 an implication reduces to false on the left hand side, the right hand side is meaningless.

\subsection*{Example: inconsistent assertions}

In the following example, it is quite simple to see the inconsistency of the assertions. When systems are large and more
convoluted, inconsistencies can be more difficult to spot. The simpleness of the example is to get to the heart of the matter.

\begin{lstlisting}

method inconsistent(a : bool, b : bool) returns (result : bool)
    requires a && b
    ensures a ==> b
    ensures a && !b

\end{lstlisting}

We are trying to ensure that \(b\) is both true and untrue. This is inconsistent. All clauses must be satisfiable
at the same time for a contract to be consistent. As stated previously, with an implementation attached to this contract,
Dafny would catch the issue and report that the implementation does not agree with the contract. What Dafny would not
tell the developer is that the implementation can {\it never} satisfy the specification. In systems with complicated
contracts, a developer could spend time trying to fix an implementation that has no hope of satisfying an inconsistent
contract.

\subsection*{Example: inconsistent requirements}

Consider the following example:

\begin{lstlisting}

method BinarySearch(a: array<int>, key: int) returns (index: int)
    requires forall i, j | 0 <= i < a.Length && 0 <= j < a.Length
        :: i <= j ==> a[i] < a[j]
    ensures 0 <= index < a.Length ==> a[index] == key
    ensures index < 0 ==> forall k :: 0 <= k < a.Length ==> a[k] != key

\end{lstlisting}

The \(requires\) clause above has an error. The intent of the clause is to require a sorted array \(a\) as input.
However, the claim that \(i <= j \implies a[i] < a[j]\) is false. If the indeces are the same, then \(a\) at that
index has to be the same. Because the relationship between requires and ensures is an implication, this false requirement
makes any ensures clause provable. Dafny currenly will not catch this error, with or without an implementation attached to
the contract.

\section{Unconstrained outputs}

When writing a contract, the whole point is to be able to talk about the behavior of the system being specified.
Every system has inputs (whether implicit or explicit) and outputs, and specifications tell us what these pieces need
to look like. Let’s look at the idea of unconstrained outputs with an example:

\begin{lstlisting}

method absolute_value(n : int) returns (result : int)
    ensures n < 0 ==> result == -1 * n
{
    if (n < 0) {
        return -1 * n;
    } else {
        return n;
    }
}

\end{lstlisting}

We can easily see that this implementation of a method that returns the absolute value of an integer is correct. A developer may be
tempted to only ensure something about the case that changes the value of \(n\). However, in the case that \(n >= 0\), the value
of \(result\) is totally unconstrained. If we changed the implementation to return a hard-coded value for that case, everything
would still prove out:

\begin{lstlisting}

method absolute_value(n : int) returns (result : int)
    ensures n < 0 ==> result == -1 * n
{
    if (n < 0) {
        return -1 * n;
    } else {
        return 142;
    }
}

\end{lstlisting}

Not accounting for every possible value of outputs causes portions of an implementation to not affect verification at all. In
most cases, it is important to understand the conditions lead to all possible outputs. This is another case where a
developer could be led to believe that an implementation that proves out is correct, when in reality, the specification is
simply not strong enough.

\section{Redundancy}

A software contract is redundant if there are unnecessary components in it. More formally, given a contract
\(\Phi = \{\varphi_{1}, \varphi_{2}, ..., \varphi_{n}\}\), if \(\{\varphi_{2} \land ... \land \varphi_{n}\} \implies \varphi_{1}\), then
\(varphi_{1}\) is redundant - it is already implied by the other formulae!

While redundancy does not cause direct errors, it can lead to convoluted specifications that are more error prone. Specifications should be as
clear and simple as possible, so the removal of redundant properties allows for more streamlined specifications and takes away some chance
for misunderstanding or error.

Consider the following example:

\begin{lstlisting}

method redundant(x : int) returns (result : int)
    requires 0 <= x < 100
    requires x != -1 && x != 50

\end{lstlisting}

In this simple example, we can see that the developer wants to specify that the input must be between 0 and 100, and cannot be
-1 or 50. The \(x \neq -1\) is redundant. \(0 \leq x < 100 \implies x \neq -1\) already.

\section{Approach}

We will use the Z3 SMT solver as the main tool for detecting each above pitfall. As previosly stated,
this work will be implemented for Dafny, but the techniques used could easily be used elsewhere.

\subsection*{Detecting inconsistencies}

Detecting inconsistencies in specifications is as simple as determining whether the contract as a whole is satisfiable.
We need only to take the conjunction of each subformula, and ask Z3 if there is a satifying assignment. If Z3 tells us that the
formula is unsatisfiable, we know that we have an inconsistency. However, we want more information than just that. We
want to know what parts of the contract are inconsistent with other parts. This set of inconsistent formulas could be
nontrivial, as there could exisit multiple subformulas that do not agree with other subformulas. We want to report all
such combinations to the developer.

In order to acheive this, we draw from Barnat's \cite{barnat2016analysing} work on inconsistency detection.
To find the combinations of consistent formulae and the combinations of inconsistent formulae in a contract
\(\Phi = \{\varphi_{1}, \varphi_{2}, ..., \varphi_{n}\}\), we start with the set \(\{\varphi_{1} \land \varphi_{2} \land ... \land \varphi_{n}\} \)
and the sets that contain each individual formula: \(\{\varphi_{1}\}, \{\varphi_{2}\}, ... \{\varphi_{n}\} \), and check
the consistency of each set. If any of the individual formulae are inconsistent alone, then \(\Phi\) will also be inconsistent.
We then go through a processes of removing formulae from \(\Phi\), and adding formulae to consistent individual sets.

We will also check for inconsistency in clauses that have implications, to discover any vacuity in requirements.

\subsection*{Detecting unconstrained output}

Unconstrained output has to do with the domain of the output variables of a method. If a contract describes the entire
domain of the variable at hand, then every possible value of that variable is accounted for. The process of detecting
unconstrained output can also be accomplished by using an SMT solver. Jaffe et al. \cite{jaffe1989completeness} discuss how
an output is "complete", or in our terms, constrained, if the disjunction of all final assertions about that output is a
tautology. For example, if there is a clause that ensures that given a certain condition, the output \(x < 5\), in order
for {\it x} to be fully constrained, we would need a condition, or set of conditions, that ensure that the output \(x \geq 5\).
Thus \(x < 5 \lor x \geq 5\), which always evaluates to true.

Our approach for detecting unconstrained outputs will be based on this insight:
\break

\begin{algorithmic}

\ForAll{$o \in \mathcal Outputs $}
    \State $relevantClauses \gets c_{1}, c_{2}, ..., c_{n}$ \Comment{Relevant clauses are those that mention o}
    \If{$c_{1} \lor c_{2} \lor ... \lor c_{n}$ is true}
        \State $result \gets constrained$
    \Else
        \State $result \gets unconstrained$
    \EndIf
\EndFor

\end{algorithmic}

Sending the disjunction of the relevant clauses to Z3 will tell us whether each output is sufficiently constrained.
We are also interested in finding a way to return the part of the domain for each unconstrained output that
is not covered.

\subsection*{Detecting redundancy}

Redundancy also draws from Barnat's work. Given a contract \(\Phi = \{\varphi_{1}, \varphi_{2}, ..., \varphi_{n}\}\),
\(\varphi_{1}\) is not a redundant formula if \(\{\not \varphi_{1} \land \varphi_{2} \land ... \land \varphi_{n}\}\)
is satisfiable, because \(\{\varphi_{2} \land ... \land \varphi_{n}\} \not\implies \varphi_{1}\). So, we can check
each formula to see if it is redundant with respect to the rest of the formulae.

\section{Deliverables}

The main deliverable of this work will be a set of methods that use an SMT solver to determine the quality of software
contracts. The thesis will provide algorithms, as well as the specifics of the implementation in Dafny in order to make
extensions of this work accessible in the future. These methods will be implemented for Dafny, but could easily be
applied to other specification-based languages. The tool we create will be an option in Dafny that a developer can
use to check their specifications of any given function or method. Any potential issues will be reported back to the
developer.

\section{Timeline}

% \bibliographystyle{alpha}
% \bibliography{sample}
% \bibliographystyle{agsm}
\bibliography{References}

\end{document}